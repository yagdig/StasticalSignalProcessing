\documentclass[fontset=windows]{article}
\usepackage[margin=1in]{geometry}%设置边距,符合Word设定
\usepackage[UTF8]{ctex}
\usepackage{setspace}
\usepackage{amsmath}
\usepackage{amssymb}
\numberwithin{figure}{section}
\usepackage{array}
\usepackage{lipsum}
\usepackage{float}
\usepackage{graphicx}%插入图片
\usepackage[dvipsnames]{xcolor}
\usepackage{authblk}
\usepackage{listings,matlab-prettifier}
\lstset{
	language=Matlab, % 设置代码语言为Matlab
    basicstyle=\ttfamily, % 设置字体为等宽字体
    numbers=left, % 行号在左边显示
    numberstyle=\tiny, % 设置行号字体大小
    stepnumber=1, % 行号递增步长
    numbersep=5pt, % 行号到代码的距离
    backgroundcolor=\color{gray!10}, % 设置代码的背景颜色
    showspaces=false,
    showstringspaces=false,
    showtabs=false,
    frame=single, % 设置代码框
    rulecolor=\color{black},
    tabsize=2,
    breaklines=true,
    breakatwhitespace=true,
    title=\lstname,
	keywordstyle=\bfseries\color{NavyBlue},
	morekeywords={var,};
	emphstyle=\bfseries\color{Rhodamine}, % 强调词样式设置
    commentstyle=\itshape\color{black!50!white}, % 设置注释样式,斜体,浅灰色
    stringstyle=\bfseries\color{PineGreen!90!black}, % 设置字符串样式
	columns=flexible,}
\graphicspath{{Figures2/}}%文章所用图片在当前目录下的 Figures目录

\usepackage{hyperref} % 对目录生成链接,注:该宏包可能与其他宏包冲突,故放在所有引用的宏包之后
\hypersetup{colorlinks = true,  % 将链接文字带颜色
	linkcolor=black, % 将链接文字黑色
	bookmarksopen = true, % 展开书签
	bookmarksnumbered = true, % 书签带章节编号
	} % 作者
\bibliographystyle{plain}% 参考文献引用格式

\renewcommand{\contentsname}{\centerline{目录}} %经过设置word格式后,将目录标题居中

\title{\heiti\zihao{2} 《统计信号处理》讨论题零:随机变量函数}
\author{杨 鼎}
\date{}

\begin{document}
\maketitle
\thispagestyle{empty}

%\begin{abstract}
%	\lipsum[2]
%\end{abstract}

%\tableofcontents
%\setcounter{page}{0}
%\newpage

\section{第一题}
已知随机变量X的概率密度为\(f_X(x)\),且\(Y=X\),则X和Y的联合概率密度\(f_{XY}(x,y)=?\)

随机变量X的概率密度函数为\(f_X(x)\),则其概率分布函数为
\begin{align*}
    F_X(x)=P(X\leq x)=\int_{-\infty}^{x} f_X(t)dt
\end{align*},
由\(Y=X\)可知,Y的概率分布函数\(F_Y(y)=\int_{-\infty}^{y}f_Y(t)dt\).

X和Y的联合概率分布函数为\(F_{XY}(x,y)=P(X\leq x,Y\leq y)\),由于\(X=Y\),
\begin{align*}
    P(X\leq x,Y\leq y)
     & =P(X\leq \min(x,y),Y\leq \min(x,y)) \\
     & =P(X\leq \min(x,y))                 \\
     & =P(Y\leq \min(x,y))
\end{align*}



于是X和Y的联合概率分布函数为
\begin{align*}
    F_{XY}(x,y)=P(X\leq \min(x,y))=\int_{-\infty}^{\min(x,y)} f_{X}(t)dt
\end{align*}

注意到\(\min(x,y)\)可以写为\(\frac{x+y-|x-y|}{2}\),X和Y的联合概率密度函数
\begin{align*}
    f_{XY}(x,y)
     & =\frac{\partial^2}{\partial x \partial y}F_{XY}(x,y)=\frac{\partial^2}{\partial x \partial y} \int_{-\infty}^{\min(x,y)} f_{X}(t)dt \\
     & =\frac{\partial^2}{\partial x \partial y} \int_{-\infty}^{\frac{x+y-|x-y|}{2}} f_{X}(t)dt                                           \\
     & =\frac{\partial}{\partial x}\left[f_X(\frac{x+y-|x-y|}{2})\cdot\frac{1}{2}(1+\frac{|x-y|}{x-y}) \right]                             \\
     & =\frac{\partial}{\partial x}\left[f_X(\frac{x+y-|x-y|}{2})u(x-y) \right]                                                            \\
     & =\frac{\partial}{\partial x}\left[f_X(y)u(x-y) \right]                                                                              \\
     & =f_X(y)\delta(x-y)
\end{align*}
或
\begin{align*}
    f_{XY}(x,y)
     & =\frac{\partial^2}{\partial y \partial x}F_{XY}(x,y)                                                    \\
     & =\frac{\partial^2}{\partial y \partial x} \int_{-\infty}^{\frac{x+y-|x-y|}{2}} f_{X}(t)dt               \\
     & =\frac{\partial}{\partial y}\left[f_X(\frac{x+y-|x-y|}{2})\cdot\frac{1}{2}(1-\frac{|x-y|}{x-y}) \right] \\
     & =\frac{\partial}{\partial y}\left[f_X(\frac{x+y-|x-y|}{2})u(y-x) \right]                                \\
     & =\frac{\partial}{\partial y}\left[f_X(x)u(y-x) \right]                                                  \\
     & =f_X(x)\delta(y-x)
\end{align*}

\section{第二题}
已知随机变量X的概率密度为\(f_X(x)\),且\(Y=g(X)\),其中\(g(\cdot)\)为确定性的函数,
则X和Y的联合概率密度\(f_{XY}=?\)

随机变量X的概率密度函数为\(f_X(x)\),则其概率分布函数为
\begin{align*}
    F_X(x)=P(X\leq x)=\int_{-\infty}^{x} f_X(t)dt
\end{align*}

X和Y的联合概率密度函数为\(f_{XY}(x,y)=f_{Y|X}(y|x)f_X(x)\),\(f_X(x)\)已知,下面计算\(f_{Y|X}(y|x)\)。

由于\(Y=g(X)\),所以若\(y<g(x),P(Y<y|x)=0\),若\(y>g(x),P(Y<y|x)=1\),即
\begin{align*}
    F_{Y|X}(y|x) & =P(Y<y|x)  \\
                 & =\left\{
    \begin{matrix}
        1 & ,y\geq  g(x) \\
        0 & ,y< g(x)
    \end{matrix}
    \right.                   \\
                 & =u(y-g(x))
\end{align*}
所以
\begin{align*}
    f_{Y|X}(y|x)=\frac{\partial F_{Y|X}(y|x)}{\partial y}=\delta(y-g(x))
\end{align*}

于是\(f_{XY}(x,y)=f_{Y|X}(y|x)f_X(x)=f_X(x)\delta(y-g(x))\)。

\bibliography{books}
\end{document}