\documentclass[fontset=windows]{article}
\usepackage[margin=1in]{geometry}%设置边距,符合Word设定
\usepackage[UTF8]{ctex}
\usepackage{setspace}
\usepackage{amsmath}
\usepackage{amssymb}
\numberwithin{figure}{section}
\usepackage{array}
\usepackage{lipsum}
\usepackage{float}
\usepackage{graphicx}%插入图片
\usepackage[dvipsnames]{xcolor}
\usepackage{authblk}
\usepackage{listings,matlab-prettifier}
\lstset{
	language=Matlab, % 设置代码语言为Matlab
    basicstyle=\ttfamily, % 设置字体为等宽字体
    numbers=left, % 行号在左边显示
    numberstyle=\tiny, % 设置行号字体大小
    stepnumber=1, % 行号递增步长
    numbersep=5pt, % 行号到代码的距离
    backgroundcolor=\color{gray!10}, % 设置代码的背景颜色
    showspaces=false,
    showstringspaces=false,
    showtabs=false,
    frame=single, % 设置代码框
    rulecolor=\color{black},
    tabsize=2,
    breaklines=true,
    breakatwhitespace=true,
    title=\lstname,
	keywordstyle=\bfseries\color{NavyBlue},
	morekeywords={var,};
	emphstyle=\bfseries\color{Rhodamine}, % 强调词样式设置
    commentstyle=\itshape\color{black!50!white}, % 设置注释样式,斜体,浅灰色
    stringstyle=\bfseries\color{PineGreen!90!black}, % 设置字符串样式
	columns=flexible,}
\graphicspath{{Figures2/}}%文章所用图片在当前目录下的 Figures目录

\usepackage{hyperref} % 对目录生成链接,注:该宏包可能与其他宏包冲突,故放在所有引用的宏包之后
\hypersetup{colorlinks = true,  % 将链接文字带颜色
	linkcolor=black, % 将链接文字黑色
	bookmarksopen = true, % 展开书签
	bookmarksnumbered = true, % 书签带章节编号
	} % 作者
\bibliographystyle{plain}% 参考文献引用格式

\renewcommand{\contentsname}{\centerline{目录}} %经过设置word格式后,将目录标题居中

\title{\heiti\zihao{2} 《统计信号处理》讨论题二:贝叶斯估计}
\author{杨 鼎}
\date{}

\begin{document}
\maketitle
\thispagestyle{empty}

%\begin{abstract}
%	\lipsum[2]
%\end{abstract}

%\tableofcontents
%\setcounter{page}{0}
%\newpage

贝叶斯估计和非贝叶斯估计的区别是什么?贝叶斯估计的先验信息如何获得?对于高斯白噪声中指数信号参数估计问题
\begin{align*}
    z_n=Ar^n+w_n,\quad n=0,1,\cdots,N-1
\end{align*}
其中\(A\sim N(0,\sigma^2),w_n\overset{i,i,d}{\sim}\),A与\(w_n\)相互独立;\(0<r<1,\sigma^2>0,\sigma^2_A>0\)均为已知参数。试求参数A的最小均方估计和最大后验概率估计,并分析二者的关系。若对参数A没有更多额先验信息时可如何处理?

\section*{1.贝叶斯估计和非贝叶斯估计的区别是什么?贝叶斯估计的先验信息如何获得?}

贝叶斯估计和非贝叶斯估计的区别是有无先验信息。

贝叶斯估计的先验信息通常是建立在物理约束的基础上;也可以从数据估计先验,称为经验贝叶斯。

\section*{2.对于高斯白噪声中指数信号参数估计问题
  \begin{align*}
      z_n=Ar^n+w_n,\quad n=0,1,\cdots,N-1
  \end{align*}
  其中\(A\sim \mathcal{N}(0,\sigma^2),w_n\overset{i,i,d}{\sim}\mathcal{N}(0,\sigma^2)\),A与\(w_n\)相互独立;\(0<r<1,\sigma^2>0,\sigma^2_A>0\)均为已知参数。试求参数A的最小均方估计和最大后验概率估计,并分析二者的关系。若对参数A没有更多额先验信息时可如何处理?}

\subsection*{(1)参数A的最小均方估计}
观测模型用矢量形式表示
\begin{align*}
    \mathbf{Z}=A\mathbf{H}+\mathbf{w}
\end{align*}
其中
\begin{align*}
    \mathbf{Z} & =\begin{bmatrix}z_0&z_1&\cdots&z_{N-1}\end{bmatrix}^T \\
    \mathbf{w} & =\begin{bmatrix}w_0&w_1&\cdots&w_{N-1}\end{bmatrix}^T \\
    \mathbf{H} & =\begin{bmatrix}r_0&r_1&r_2&\cdots&r^{N-1}\end{bmatrix}^T
\end{align*}
由于\(E(A)=0,E(\mathbf{w})=\mathbf{0},E(\mathbf{Z})=\mathbf{0}\),则
\begin{align*}
    \mathbf{C}_{A\mathbf{Z}}
     & =E[A(\mathbf{Z}-E(\mathbf{Z}))^T] \\
     & =E[A(A\mathbf{H}+\mathbf{w})^T]   \\
     & =E[A^2\mathbf{H}^T]               \\
     & =\sigma^2_A\mathbf{H}^T
\end{align*}

\begin{align*}
    \mathbf{C}_{\mathbf{Z}}
     & =E[(\mathbf{Z}-E(\mathbf{Z}))(\mathbf{Z}-E(\mathbf{Z}))^T] \\
     & =E[(A\mathbf{H}+\mathbf{w})(A\mathbf{H}+\mathbf{w})^T]     \\
     & =E[A^2\mathbf{H}\mathbf{H}^T]+E[\mathbf{w}\mathbf{w}^T]    \\
     & =\sigma^2_A\mathbf{HH}^T+\sigma^2\mathbf{I}
\end{align*}

最小MSE估计量
\begin{align*}
    \hat{A}=E(A|\mathbf{Z})
     & =E(\mathbf{A})+\mathbf{C}_{A\mathbf{Z}}\mathbf{C}_{\mathbf{Z}}^{-1}(\mathbf{Z}-E(\mathbf{Z}))                 \\
     & =\sigma_A^2\mathbf{H}^T(\sigma^2_A\mathbf{HH}^T+\sigma^2\mathbf{I})^{-1}\mathbf{Z}                            \\
     & =\frac{\sigma_A^2}{\sigma^2}\mathbf{H}^T(\frac{\sigma_A^2}{\sigma^2}\mathbf{HH}^T+\mathbf{I})^{-1}\mathbf{Z}  \\
     & =\frac{\sigma_A^2}{\sigma^2}\mathbf{H}^T(\mathbf{I}-\frac{\sigma_A^2}{\sigma^2}\mathbf{H}
    (\mathbf{I}+\frac{\sigma_A^2}{\sigma^2}\mathbf{H}^T\mathbf{H})^{-1}\mathbf{H}^T)\mathbf{Z}                       \\
     & =\frac{\sigma_A^2}{\sigma^2}\mathbf{H}^T(\mathbf{I}-\frac{\sigma_A^2/\sigma^2\mathbf{H}\mathbf{H}^T}
    {\mathbf{I}+\sigma_A^2/\sigma^2\mathbf{H}^T\mathbf{H}})\mathbf{Z}                                                \\
     & =\frac{\sigma_A^2}{\sigma^2}(\mathbf{H}^T-\frac{\sigma_A^2/\sigma^2\mathbf{H}^T\mathbf{H}\mathbf{H}^T}
    {\mathbf{I}+\sigma_A^2/\sigma^2\mathbf{H}^T\mathbf{H}})\mathbf{Z}                                                \\
     & =\frac{\sigma_A^2}{\sigma^2}(1-\frac{\sigma_A^2/\sigma^2\mathbf{H}^T\mathbf{H}}
    {1+\sigma_A^2/\sigma^2\mathbf{H}^T\mathbf{H}})\mathbf{H}^T\mathbf{Z}                                             \\
     & =\frac{\sigma_A^2}{\sigma^2}\frac{\sigma^2}{\sigma^2+\sigma^2_A+\mathbf{H}^T\mathbf{H}}\mathbf{H}^T\mathbf{Z} \\
     & =\frac{\sigma^2_A}{\sigma^2_A\frac{1-r^{2N}}{1-r^2}+\sigma^2}\sum_{n=0}^{N-1}r_nz_n
\end{align*}
其中用到了矩阵求逆引理\((\mathbf{A}+\mathbf{BCD})^{-1}=\mathbf{A}^{-1}-\mathbf{A}^{-1}\mathbf{B}(\mathbf{DA}^{-1}\mathbf{B}+\mathbf{C}^{-1})^{-1}\mathbf{DA}^{-1}\)。
\subsection*{(2)参数A的最大后验概率估计}

最大后验概率估计为\(\hat{A}_{map}=arg\underset{A}{\max}p(A|\mathbf{Z})\),由贝叶斯公式\(p(A|\mathbf{Z})=\frac{p(\mathbf{Z}|A)p(A)}{p(\mathbf{Z})}\),
最大后验概率估计等价为\(\hat{A}_{map}=arg\underset{A}{\max}p(\mathbf{Z}|A)p(A)\),已知
\begin{align*}
    p(\mathbf{Z}|A) & =(\frac{1}{2\pi \sigma^2})^{\frac{N}{2}}\exp\left\{-\frac{1}{2\sigma^2}\sum_{n=0}^{N-1}(z_n-Ar^n)^2\right\} \\
    p(A)            & =\frac{1}{\sqrt{2\pi \sigma^2_A}}\exp\{-\frac{A^2}{2\sigma^2_A}\}
\end{align*}

所以
\begin{align*}
    \ln[p(\mathbf{Z}|A)p(A)]=-\frac{N\ln (2\pi \sigma^2)+\ln(2\pi \sigma^2_A)}{2}-\frac{1}{2\sigma^2}\sum_{n=0}^{N-1}(z_n-Ar^n)^2-\frac{1}{2\sigma^2_A}A^2
\end{align*}
令\(\ln[p(\mathbf{Z}|A)p(A)]=0\),得到
\begin{align*}
    \frac{1}{\sigma^2}\sum_{n=0}^{N-1}r^n(z_n-Ar^n)=\frac{A}{\sigma^2_A}
\end{align*}
解得
\begin{align*}
    \hat{A}_{map}=\frac{\sigma^2_A}{\sigma^2_A\frac{1-r^{2N}}{1-r^2}+\sigma^2}\sum_{n=0}^{N-1}z_nr^n
\end{align*}
与最小均方估计结果相同。

若对参数A没有更多的先验信息,可以考虑用经典估计方法。

\bibliography{books}
\end{document}