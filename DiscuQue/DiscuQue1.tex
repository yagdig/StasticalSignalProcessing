\documentclass[fontset=windows]{article}
\usepackage[margin=1in]{geometry}%设置边距,符合Word设定
\usepackage[UTF8]{ctex}
\usepackage{setspace}
\usepackage{amsmath}
\usepackage{amssymb}
\numberwithin{figure}{section}
\usepackage{array}
\usepackage{lipsum}
\usepackage{float}
\usepackage{graphicx}%插入图片
\usepackage[dvipsnames]{xcolor}
\usepackage{authblk}
\usepackage{listings,matlab-prettifier}
\lstset{
	language=Matlab, % 设置代码语言为Matlab
    basicstyle=\ttfamily, % 设置字体为等宽字体
    numbers=left, % 行号在左边显示
    numberstyle=\tiny, % 设置行号字体大小
    stepnumber=1, % 行号递增步长
    numbersep=5pt, % 行号到代码的距离
    backgroundcolor=\color{gray!10}, % 设置代码的背景颜色
    showspaces=false,
    showstringspaces=false,
    showtabs=false,
    frame=single, % 设置代码框
    rulecolor=\color{black},
    tabsize=2,
    breaklines=true,
    breakatwhitespace=true,
    title=\lstname,
	keywordstyle=\bfseries\color{NavyBlue},
	morekeywords={var,};
	emphstyle=\bfseries\color{Rhodamine}, % 强调词样式设置
    commentstyle=\itshape\color{black!50!white}, % 设置注释样式,斜体,浅灰色
    stringstyle=\bfseries\color{PineGreen!90!black}, % 设置字符串样式
	columns=flexible,}
\graphicspath{{Figures2/}}%文章所用图片在当前目录下的 Figures目录

\usepackage{hyperref} % 对目录生成链接,注:该宏包可能与其他宏包冲突,故放在所有引用的宏包之后
\hypersetup{colorlinks = true,  % 将链接文字带颜色
	linkcolor=black, % 将链接文字黑色
	bookmarksopen = true, % 展开书签
	bookmarksnumbered = true, % 书签带章节编号
	} % 作者
\bibliographystyle{plain}% 参考文献引用格式

\renewcommand{\contentsname}{\centerline{目录}} %经过设置word格式后,将目录标题居中

\title{\heiti\zihao{2} 《统计信号处理》讨论题一:确定性参数估计}
\author{杨 鼎}
\date{}

\begin{document}
\maketitle
\thispagestyle{empty}

%\begin{abstract}
%	\lipsum[2]
%\end{abstract}

%\tableofcontents
%\setcounter{page}{0}
%\newpage

考虑高斯白噪声中的指数信号参数估计问题
\begin{align*}
    z_n=Ar^n+w_n,n=0,1,\dots,N-1
\end{align*}
其中\(w_n \overset{i,i,d}{\sim} \mathcal{N}(0,\sigma^2),0<r<1,\sigma^2>0\)均为已知参数。
试考虑可从哪些角度获得参数A的估计量?如何评价你所获得估计量的性能?
\section*{1.获得A的估计量}
重写为矢量形式
\begin{align*}
    \mathbf{z}=A\mathbf{H}+\mathbf{w}
\end{align*}
其中
\begin{align*}
    \mathbf{z} & =\begin{bmatrix}
        z_1 & z_2 & \dots & z_{N-1}
    \end{bmatrix}^T \\
    \mathbf{H} & =\begin{bmatrix}
        1 & r & \dots & r^{N-1}
    \end{bmatrix}^T \\
    \mathbf{w} & =\begin{bmatrix}
        w_1 & w_2 & \dots & w_{N-1}
    \end{bmatrix}^T
\end{align*}

由\(\mathbf{w} \sim \mathcal{N}(\mathbf{0},\sigma^2\mathbf{I})\)可得,PDF\(p(\mathbf{z};A)\)
\begin{align*}
    p(\mathbf{z};A)=\frac{1}{(2\pi \sigma^2)^{\frac{N}{2}}}
    \exp\left\{-\frac{1}{2\sigma^2}(\mathbf{z}-A\mathbf{H})^T(\mathbf{z}-A\mathbf{H})\right\}
\end{align*}
求一次导得到:
\begin{align*}
    \frac{\partial \ln p(\mathbf{z};A)}{\partial A}=\frac{\mathbf{H}^T}{\sigma^2}(\mathbf{z}-A\mathbf{H})
\end{align*}

显然这是一个关于未知参数A的线性模型,因此可以用多种方式获得参数A的估计。

\subsection*{(1)CRLB}

首先验证正则条件
\begin{align*}
    E\left[\frac{\partial \ln p(\mathbf{x};A)}{\partial A}\right]
    =E\left[\frac{\mathbf{H}^T}{\sigma^2}(\mathbf{z}-A\mathbf{H})\right]
    =\frac{\mathbf{H}^T}{\sigma^2}E\left[(\mathbf{z}-A\mathbf{H})\right]=0
\end{align*}
则Fisher信息为
\begin{align*}
    I(A) & =-E\left[\frac{\partial^2  \ln p(\mathbf{x};A)}{\partial A^2}\right] \\
         & =\frac{\mathbf{H}^T\mathbf{H}}{\sigma^2}                             \\
         & =\frac{\sum_{n=0}^{N-1}r^{2n}}{\sigma^2}                             \\
         & =\frac{1-r^{2N}}{(1-r^2)\sigma^2}
\end{align*}
PDF的一阶偏导数可以写为
\begin{align*}
    \frac{\partial \ln p(\mathbf{z};A)}{\partial A}
     & =\frac{\mathbf{H}^T}{\sigma^2}(\mathbf{z}-A\mathbf{H}) \\
     & =\frac{\mathbf{H}^T\mathbf{H}}{\sigma^2}
    ((\mathbf{H}^T\mathbf{H})^{-1}\mathbf{H}^T\mathbf{z}-A)   \\
     & =I(A)(g(\mathbf{z})-A)
\end{align*}
其中\(\mathbf{H}^T\mathbf{H}=\sum_{n=0}^{N-1}r^{2n}=\frac{1-r^{2n}}{1-r}\),
因此\(\hat{A}=(\mathbf{H}^T\mathbf{H})^{-1}\mathbf{H}^T\mathbf{z}\)。
方差达到CRLB,为
\begin{align*}
    var(\hat{A})=\frac{1}{I(A)}=\frac{(1-r^2)\sigma^2}{1-r^{2n}}
\end{align*}
\subsection*{(2)BLUE}

限定估计量A与数据\(\mathbf{z}\)呈线性\(A=\mathbf{a}^T\mathbf{z}\),在无偏的约束条件下,
\(E(\mathbf{z})=A\mathbf{H}\).

由于\(\mathbf{w}\sim \mathcal{N}(0,\sigma^2\mathbf{I})\),协方差矩阵\(\mathbf{C}=\sigma^2 \mathbf{I}\),
则\(\mathbf{C^{-1}}=\sigma^{-2}\mathbf{I}\)。

BLUE为
\begin{align*}
    \hat{A}
     & =\frac{\mathbf{H}^T \mathbf{C}^{-1}\mathbf{z}}{\mathbf{H}^T\mathbf{C}^{-1}\mathbf{H}} \\
     & =(\mathbf{H}^T \mathbf{H})^{-1}\mathbf{H}\mathbf{z}
\end{align*}
方差
\begin{align*}
    var(\hat{A})
     & =\frac{1}{\mathbf{H}^T \mathbf{C}^{-1} \mathbf{H}} \\
     & =\frac{(1-r^2)\sigma^2}{1-r^{2n}}
\end{align*}

\subsection*{(3)MLE}
求解似然方程
\begin{align*}
    \frac{\mathbf{H}^T}{\sigma^2}(\mathbf{z}-A\mathbf{H})=\mathbf{0}
\end{align*}
解得
\begin{align*}
    \hat{A}=(\mathbf{H}^T\mathbf{H})^{-1}\mathbf{H}^T \mathbf{z}
\end{align*}

验证二阶导数
\begin{align*}
    \frac{\partial^2 \ln p(\mathbf{z};A)}{\partial A^2}
    =-\frac{\mathbf{H}^T\mathbf{H}}{\sigma^2}<0
\end{align*}
因此A的最大似然估计为\(\hat{A}=(\mathbf{H}^T\mathbf{H})^{-1}\mathbf{H}^T \mathbf{z}\)。

\subsection*{(4)LSE}
误差指标函数
\begin{align*}
    \mathbf{J}(A)=(\mathbf{z}-A\mathbf{H})^T(\mathbf{z}-A\mathbf{H})
\end{align*}
令其梯度\(\frac{\partial\mathbf{J}(A)}{\partial A}=0\)得到,
LSE为\(\hat{A}=(\mathbf{H}^T\mathbf{H})^{-1}\mathbf{H}^T\mathbf{z}\)。

\section*{2.如何评价你所获得估计量的性能}

以上四种方法求得的A的估计量\(\hat{A}\)均相同,且均是无偏的。

在(1)中,由于达到了CRLB,因此它是有效估计量,并且它也是MVU估计量。

根据Neyman-Fisher因子分解定理,由于PDF可以分解为
\begin{align*}
    p(\mathbf{z};A)
     & =\frac{1}{(2\pi \sigma^2)^{\frac{N}{2}}}
    \exp\left\{-\frac{1}{2\sigma^2}(\mathbf{z}-A\mathbf{H})^T(\mathbf{z}-A\mathbf{H})\right\} \\
     & =\frac{1}{(2\pi \sigma^2)^{\frac{N}{2}}}
    \exp\left\{-\frac{1}{2\sigma^2}(A^2\mathbf{H}^T\mathbf{H}-2A\mathbf{z}^T\mathbf{H})\right\}
    \cdot\exp\left\{-\frac{\mathbf{z}^T\mathbf{z}}{2\sigma^2})\right\}                        \\
     & =g(T(\mathbf{z}),A)h(\mathbf{z})
\end{align*}
因此\(T(\mathbf{z})=\mathbf{H}^T\mathbf{z}\)是A的充分统计量,
\(\hat{A}=(\mathbf{H}^T\mathbf{H})^{-1}T(\mathbf{z})\)也是A的充分统计量。

\bibliography{books}
\end{document}