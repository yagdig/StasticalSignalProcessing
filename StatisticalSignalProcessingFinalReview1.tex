\documentclass[fontset=windows]{article}
\usepackage[margin=1in]{geometry}%设置边距,符合Word设定
\usepackage[UTF8]{ctex}
\usepackage{setspace}
\usepackage{amsmath}
\numberwithin{figure}{section}
\usepackage{array}
\usepackage{lipsum}
\usepackage{float}
\usepackage{graphicx}%插入图片
\usepackage[dvipsnames]{xcolor}
\usepackage{authblk}
\usepackage{listings,matlab-prettifier}
\lstset{
	language=Matlab, % 设置代码语言为Matlab
    basicstyle=\ttfamily, % 设置字体为等宽字体
    numbers=left, % 行号在左边显示
    numberstyle=\tiny, % 设置行号字体大小
    stepnumber=1, % 行号递增步长
    numbersep=5pt, % 行号到代码的距离
    backgroundcolor=\color{gray!10}, % 设置代码的背景颜色
    showspaces=false,
    showstringspaces=false,
    showtabs=false,
    frame=single, % 设置代码框
    rulecolor=\color{black},
    tabsize=2,
    breaklines=true,
    breakatwhitespace=true,
    title=\lstname,
	keywordstyle=\bfseries\color{NavyBlue},
	morekeywords={var,};
	emphstyle=\bfseries\color{Rhodamine}, % 强调词样式设置
    commentstyle=\itshape\color{black!50!white}, % 设置注释样式,斜体,浅灰色
    stringstyle=\bfseries\color{PineGreen!90!black}, % 设置字符串样式
	columns=flexible,
}
\graphicspath{{Figures1/}}%文章所用图片在当前目录下的 Figures目录

\usepackage{hyperref} % 对目录生成链接,注:该宏包可能与其他宏包冲突,故放在所有引用的宏包之后
\hypersetup{colorlinks = true,  % 将链接文字带颜色
	linkcolor=black, % 将链接文字黑色
	bookmarksopen = true, % 展开书签
	bookmarksnumbered = true, % 书签带章节编号
	} % 作者
\bibliographystyle{plain}% 参考文献引用格式

\renewcommand{\contentsname}{\centerline{目录}} %经过设置word格式后,将目录标题居中

\title{\heiti\zihao{2} 《统计信号处理》期末复习}
\author{杨\ 鼎}
\date{}

\begin{document}
\maketitle
\thispagestyle{empty}

%\begin{abstract}
%	\lipsum[2]
%\end{abstract}

\tableofcontents
\setcounter{page}{0}
\newpage

\section{第一题}
设观测\(z_n\overset{i,i,d}{\sim} \mathcal{N} (\mu, \sigma^2),n=0,1,2,\cdots,N-1\),其中\(\mu\)已知,\(\sigma>0\)为确定性位置参数。试考虑如下问题:

\subsection*{(1)求\(\sigma\)的最大似然估计及克拉美-罗下限。}

\subsection*{(2)是否存在\(\sigma\)的有效估计量?若存在,试给出该估计量;若不存在,试说明原因。}

\section{第二题}
设某雷达目标ode散射界面(RSC)服从指数分布(例如Swerling I型目标),利用单个脉冲对目标进行探测时,回波幅度服从瑞利分布。单个脉冲回波信号的观测样本可以写为
\begin{align*}
	z_n=As_n+w_n\quad ,n=0,1,2,\cdots, N-1
\end{align*}
其中幅度A的概率密度函数为
\begin{align*}
	p(A)=\left\{
	\begin{matrix}
		\frac{A}{A_0^2} \exp\{-\frac{A^2}{2A^2_0} \} & ,A \geq 0 \\
		0                                            & ,A<0
	\end{matrix}
	\right.
\end{align*}
\(w_n\overset{i,i,d}{\sim}\mathcal{N}(0,\sigma^2),\sigma^2\)已知,\(\overline{\sigma}=2A^2_0\)表示目标散射截面积的平均值(已知量),\(s_n\)为已知的信号波形。试根据观测样本求回波幅度A的最大后验估计。


\section{第三题}
在高斯白噪声中观测正选信号,观测模型为
\begin{align*}
	z_n=A\cos\frac{\pi}{3}n+w_n\quad n=0,1,\cdots,N-1
\end{align*}
其中\(A\sim \mathcal{N}(0,\sigma^2_A),w_n\overset{i,i,d}{\sim}\mathcal{N}(0,\sigma^2_w)\),且A与\(w_n\)相互独立。

\subsection*{(1)记\(\mathbf{h}=\begin{bmatrix}
		\cos(\frac{\pi}{3}\cdot 1) & \cos(\frac{\pi}{3}\cdot 2) & \cdots \cos(\frac{\pi}{3}\cdot (N-1))
	\end{bmatrix}^T\),试求幅度A的线性最小均方估计(可用\(\mathbf{h}\)表示)}

\subsection*{若\(N=2,\sigma^2_A=1,\sigma^2_w=2,z_0=1,z_1=\frac{1}{2}\),求幅度A的线性最小均方估计的值。(提示:矩阵求逆引理\((\mathbf{A}+\mathbf{BCD})^{-1}=\mathbf{A}^{-1}-\mathbf{A}^{-1}\mathbf{B}(\mathbf{DA}^{-1}\mathbf{B}+\mathbf{C}^{-1})^{-1}\mathbf{DA}^{-1}\))}

\section{第四题}
考虑高斯白噪声中指数信号的检测问题,信号模型为
\begin{align*}
	\begin{matrix}
		\mathcal{H}_0: & z_n=w_n               \\
		\mathcal{H}_1: & z_n=Ae^{\alpha n}+w_n
	\end{matrix}\quad n=0,1,\cdots, N-1
\end{align*}
其中\(w_n\overset{i,i,d}{\sim}\mathcal{N}(0,\sigma^2_w),\alpha,\sigma^2_w\)均已知,信号幅度A未知。

\subsection*{(1)试求广义似然比检测器的判决式形式,并针对给定的虚警率\(P_{FA}\)确定判决门限和检测概率表达式,分析检测概率的极限性能;【提示:\(Q(x)=\int_{x}^{+\infty}\frac{1}{\sqrt{2\pi}}\exp\{-\frac{1}{2}x^2\}dx\)表示标准正态分布的右尾概率函数,\(Q_{\chi^2_N(\lambda)}(x)\)表示衷心参数为\(\lambda\)的N自由度非中心卡方分布右尾函数,\(Q_{\chi^2_N}(x)\)表示N自由度卡方分布右尾函数】}

\subsection*{(2)画出检测器结构图。}

\section{第五题}
考虑SAR图像中目标检测问题。如图1所示,SAR图像中的像素包含目标(target)、背景杂波(clutter)以及阴影(shadow)等三种类型。背景杂波区、阴影区、目标区像素灰度值z可用具有不同参数的瑞利分布描述,即
\begin{align*}
	p(z;\sigma^2_i)=\left\{
	\begin{matrix}
		\frac{z}{\sigma^2_i}\exp\{-\frac{z^2}{2\sigma^2_i}\} & z\geq0 \\
		0                                                    & z<0
	\end{matrix},i=0,1,2
	\right.
\end{align*}
其中\(\sigma^2_0,\sigma^2_1,\sigma^2_2\)分别表示背景杂波区、阴影区和目标区像素灰度值的分布参数,\(\sigma^2_2>\sigma^2_0>\sigma^2_1>0\)为已知参数。目标检测的基本任务是根据给定像素的灰度值z判断该像素所在的区域。

\subsection*{(1)根据上述描述,针对像素的灰度值观测z建立假设模型;}

\subsection*{(2)假定三种假设的先验概率均相等,且\(\sigma^2_2=10\sigma^2_1,\sigma^2_0=2\sigma^2_1\),试求最小总错误概率准则下的检验判决表达式;}

\subsection*{(3)在第二问的条件下,试求将背景杂波区的像素误判为目标像素区像素的概率。(提示:可以利用函数的单调性判断对数似然函数之间的相对大小关系)}


\bibliography{books}
\end{document}